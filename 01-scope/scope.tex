Skip to content
 
Search or jump to…

Pull requests
Issues
Marketplace
Explore
 
@spaceiling 
0
0 1 dmitry-kabanov/stoch-num-2019-final-project Private
 Code  Issues 0  Pull requests 0  Projects 0  Security  Insights
stoch-num-2019-final-project/01-scope/scope.tex
@dmitry-kabanov dmitry-kabanov Add update scope, v5
785d51a 15 hours ago
224 lines (206 sloc)  9.44 KB
    
\documentclass[a4paper, 12pt, parskip]{scrartcl}
\usepackage{amsmath}
\usepackage{fourier}
\usepackage{nth}

\title{\vspace{-0.5cm}Parameter estimation\\of partial differential
        equations\\via neural networks}
\author{Alexander Glushko, Dmitry I.\ Kabanov}
%\date{23 June 2019}

\newcommand{\Data}{\vec{D}}
\newcommand{\DataExt}{\widetilde{\vec{D}}}
\newcommand{\MSE}{\ensuremath{\text{MSE}}}
\newcommand{\T}{\ensuremath{\text{T}}}
\renewcommand{\vec}[1]{\boldsymbol{#1}}
\newcommand{\VTheta}{\ensuremath{\vec{\theta}}}
\newcommand{\VLambda}{\ensuremath{\vec{\lambda}}}
\DeclareMathOperator*{\argmin}{arg\,min}
\newcommand{\R}{\mathbb R}
\newcommand{\UNN}[1][\text{NN}]{u_{#1}}
\newcommand{\FNN}[1][\text{NN}]{f_{#1}}
\newcommand{\NonlinOp}{\mathcal N\!}

\begin{document}
\maketitle

In this work, we partially reproduce and extend the work of Raissi, Perdikaris,
and Karniadakis~\cite{raissi2017pinnII} on applying neural networks to the
problem of parameter estimation of partial differential equations from given
observations.

\textbf{Problem setup}

We are given data $\Data = \{t_i, x_i, u_i\}$, $i=1, \dots, N$, that
are observed from the solution of partial differential equation of the form
\begin{equation}
    \label{eq:pde}
    u_t + \mathcal N\!(u; \VLambda) = 0,
\end{equation}
where  $u=u(x, t)$ is the solution of the equation,
$\NonlinOp(u; \VLambda)$ a nonlinear algebraic-differential operator,
$\VLambda$ the vector of unknown parameters, subscript $t$ denotes
differentiation with respect to time.
The solution $u(x,t)$ is given on the subset of $\R \times \R$ and the collected
observations are distributed uniformly on this subset.
The goal is to estimate $\VLambda$ from the observations $\Data$.

By Bayes' rule, the optimal value of $\VLambda$ is found through
maximization of the posterior distribution \cite{sivia2006data}
\begin{equation}
    \rho( \VLambda | \Data ) \propto
    \rho( \Data | \VLambda ) \times \rho( \VLambda ).
\end{equation}

We assume that the observed data are given by the exact solution plus some
additive uncorrelated Gaussian noise
\begin{equation}
    u_i = u(x_i, t_i; \VLambda) + \epsilon_i, \quad i=1, \dots, N,
\end{equation}
where $\epsilon_i \sim N(0, \sigma^2)$, and we emphasize dependence of the exact
solution $u(x, t; \VLambda)$ on the parameter vector $\VLambda$.
%\begin{align}
%    \rho(u_1,\dots,u_N | \VTheta, \sigma) &\propto \frac{1}{\sigma^N}
%    \exp \left(
%        -\frac{\sum_{i=1}^N\left(u_i-u(x_i,t_i; \VTheta)\right)^2}{2\sigma^2}
%    \right). \\
Furthermore, we assign uninformative prior for $\VLambda$
\begin{equation}
    \rho(\vec{\lambda}) = \text{const} \quad \text{ for all } \vec{\lambda},
\end{equation}
%We assume that all observations are independent from each other and that $u_i$
%and $f_i$, $i=1,\dots,N$, are independent as well and that the observations
%are Gaussian processes:
so that, in principle,
the problem of finding $\VLambda$ is a nonlinear unconstrained
optimization problem
\begin{equation}
    \label{eq:optim-ideal}
    \argmin_{\VLambda} \quad 
    \log \sum_{i=1}^{N} \big[ u_i - u(x_i, t_i; \VLambda) \big]^2,
\end{equation}
where we marginalized the likelihood treating noise variance $\sigma^2$ as a~
nuisance parameter~\cite[section~8.2]{sivia2006data}. In the following we omit
$\log$ function as its monotonic and we are interested in the optimal value of
$\VLambda$, not the value of the objective function.

Solution of the optimization problem~\eqref{eq:optim-ideal} requires multiple
solutions of Eq.~\eqref{eq:pde}, which can be expensive.
To alleviate this, we follow an approach proposed in \cite{raissi2017pinnII}.
We replace the exact
model $u(x_i, t_i; \VLambda)$ with a surrogate, where the surrogate model is a
feedforward neural network \cite{goodfellow2016deep}
\begin{equation}
\UNN(x, t; \vec{\theta}) = g_L \circ g_{L-1} \circ \dots \circ g_1,
\end{equation}
where
\[
    g_\ell(z; \VTheta_\ell) = \sigma (W_\ell z + b_\ell), \quad \ell = 1,\dots,L,
\]
with $L$ is the number of layers in the network, with layers 0 and $L$ being
input and output layers, respectively, and layers from 1 to $L-1$ being hidden
layers, $\sigma$ being a nonlinear activation function applied
componentwise. In this work, we plan to use \( \sigma(z) = \tanh (z) \).
The neural-network parameter $\VTheta$ contains the components of matrices
$W_\ell \in \R^{n_{\ell}\times n_{\ell-1}}$ and bias vectors
$b_\ell \in \R^{n_\ell}$, where $n_\ell$ denotes the width of the
$\ell$\textsuperscript{th} layer.
The hyperparameters of the neural network, $L$ (the number of the layers) and
the width of each layer, are to be determined later during the course of the
project.

We assume that the discrepancy between the exact solution $u$ and the
neural-network approximation $\UNN$ is a Gaussian noise, such that the
likelihood function is not affected.
Furthermore, to ensure that $\UNN( x, t; \VTheta)$
is close to the exact solution of Eq.~\eqref{eq:pde}, we, strictly speaking,
are supposed to formulate the problem of estimating of the unknown parameters
$\VLambda$ as the following constrained optimization problem:
\begin{subequations}
\label{eq:optim}
\begin{align}
    &\argmin_{\VLambda, \VTheta} \quad \ \ 
        \sum_{i=1}^N \big[u_i - \UNN(x_i, t_i; \VTheta)\big]^2  \\
    &\text{subject to } \ \ \UNN[\text{NN}, t]  + \NonlinOp(\UNN; \VLambda) = 0.
\end{align}
\end{subequations}
However, the equality constraint is difficult to satisfy exactly as
$u_{\text{NN}}$ is just an approximation of the exact solution of
Eq.~\eqref{eq:pde}.
Therefore, we relax the optimization problem in the following way.
We introduce a secondary neural network
\begin{equation}
    \FNN(x, t; \VLambda, \VTheta) =
        u_{\text{NN}, t} + \NonlinOp(u_{\text{NN}}; \VLambda),
\end{equation}
where we plug the neural network $\UNN$ into Eq.~\eqref{eq:pde} and
require that $\FNN \approx 0$.
Then both networks are trained simultaneously (that is, their parameters are
found) from the observations $\Data$ via the relaxed version of the optimization
problem~\eqref{eq:optim}:
\begin{equation}
    \argmin_{\VLambda, \VTheta}
    \sum_{i=1}^N \big[ u_i - \UNN(x_i, t_i; \VTheta)\big ]^2
    +\gamma \sum_{i=1}^N \big[ \FNN(x_i, t_i; \VLambda, \VTheta) \big]^2,
\end{equation}
where $\gamma$ is an extra hyperparameter that controls the importance of the
penalty term.
This parameter is chosen via cross-validation.
We assume that the number of observations $N$ is fixed (not controlled by us)
and we choose the size of the network empirically by varying its depth until
satisfying agreement is found.

%We assume that all observations are independent from each other and that $u_i$
%and $f_i$, $i=1,\dots,N$, are independent as well and that the observations
%are Gaussian processes:
%\begin{align}
%    \rho(u_1,\dots,u_N | \VTheta, \sigma) &\propto \frac{1}{\sigma^N}
%    \exp \left(
%        -\frac{\sum_{i=1}^N\left(u_i-u(x_i,t_i; \VTheta)\right)^2}{2\sigma^2}
%    \right) \\
%    \rho(f_1,\dots,f_N | \vec{\lambda}, \tau) &\propto \frac{1}{\tau^N}
%    \exp \left(
%        - \frac{\sum_{i=1}^N f(x_i, t_i; \VLambda)^2}{2\tau^2}
%    \right),
%\end{align}
%and assigning Jeffrey's priors for $\sigma$ and $\tau$, we obtain that
%\begin{equation}
%    \rho(\tilde{\vec{D}} | \VTheta, \VLambda) \propto
%    \left( \sum_{i=1}^N (u_i - u(x_i, t_i; \VTheta))^2 \right)^{-N/2} \times
%    \left( \sum_{i=1}^N (f(x_i, t_i; \VLambda))^2 \right)^{-N/2}.
%\end{equation}
%Moreover, we assume improper flat prior distribution:
%\begin{equation}
%    \rho(\vec{\theta}, \vec{\lambda}) = \text{const} \text{ for all }
%    \vec{\theta} \text{ and } \vec{\lambda}.
%\end{equation}

\textbf{Uncertainty quantification}

Training of neural networks in \cite{raissi2017pinnII} lacks uncertainty
quantification for the found parameters, that is, only the point estimates are
provided.
In this work, we apply the bootstrap procedure \cite{Wasserman2004} to simulate
$\VLambda$ and estimate its confidence sets.

% Assuming that the noise between the observations and the model is Gaussian and
% independent for each observation and assuming no prior information on the
% parameters \(\vec{\theta}\) and \(\lambda\), the training procedure is defined by
% minimizing the mean square error (MSE)
% \begin{equation}
%     \MSE = \MSE_u + \MSE_f,
% \end{equation}
% where
% \begin{equation}
%     \MSE_u = \frac{1}{N}
%         \sum_{i=1}^{N} \left[ u\left(t_i, x_i\right) - u_i \right]^2, \quad
%     \MSE_f = \frac{1}{N} \sum_{i=1}^N \left[f \left(t_i, x_i \right) \right]^2,
% \end{equation}
% which is a maximum likelihood estimation procedure.

\textbf{Applications}

\textbf{1.} As a concrete example, we consider the linear heat equation
\begin{equation}
    u_t - \lambda u_{xx} - g(t, x) = 0
\end{equation}
with one scalar sought-for parameter \(\lambda\in \R\) and given source function
\(g(t, x)\).
We generate observations from the exact solution and apply the above procedure
to identify $\lambda$. We also investigate
the performance of the parameter estimation when the observations are noisy.

\textbf{2.} As a more difficult example, we consider viscous Burgers’ equation    
\begin{equation}
    u_t + \lambda_1 u u_x - \lambda_2 u_{xx} = 0, \quad x\in[-1; 1], t\in[0, 1]
\end{equation}
with sought-for parameter \( \VLambda = \left( \lambda_1, \lambda_2 \right)^\T \).
This equation is nonlinear and serves as a prototype of the governing
equations of fluid dynamics. As in Example 1, we investigate the performance
of the parameter estimation for both clean and noisy observations.

\bibliography{scope}
\bibliographystyle{abbrv}
\end{document}
© 2019 GitHub, Inc.
Terms
Privacy
Security
Status
Help
Contact GitHub
Pricing
API
Training
Blog
About
