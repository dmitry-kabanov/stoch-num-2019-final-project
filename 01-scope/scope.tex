\documentclass[a4paper, 12pt, english, parskip]{scrartcl}
\usepackage{amsmath}
\usepackage{fourier}

\title{\vspace{-0.5cm}Parameter estimation\\of partial differential equations\\via neural networks}
\author{Alexander Glushko, Dmitry I.\ Kabanov}
\date{23 June 2019}

\newcommand{\Data}{\{D_i\}}
\newcommand{\MSE}{\ensuremath{\text{MSE}}}
\newcommand{\T}{\ensuremath{\text{T}}}

\begin{document}
\maketitle

We are interested in applying neural networks to the problem of parameter
estimation of partial differential equations from given observations.
Precisely, we are given data $\Data = \{t_i, x_i, u_i\}$,\(i=1,\dots,N\) that are
observed from the solution of partial differential equation of the form:
\begin{equation}
    u_t + \mathcal N(u) = 0,
\end{equation}
where  \(u=u(t, x)\) is the solution of the equation, subscript \(t\) denotes
differentiation with respect to time, \(\mathcal N(u; \lambda)\) is a nonlinear
algebraic-differential operator, \(\lambda\) is a vector of parameters. The goal
is to estimate \(\lambda\) from the observations \(\Data\) via simultaneous
training of neural networks \(u(x, t)\) and
\begin{equation}
    f(t, x; \lambda) = u_t + \mathcal N(u)
\end{equation}
via training procedure defined by minimizing the mean square error (MSE)
\begin{equation}
    \MSE = \MSE_u + \MSE_f,
\end{equation}
where
\begin{equation}
    \MSE_u = \frac{1}{N}
        \sum_{i=1}^{N} \left[ u\left(t_i, x_i\right) - u_i \right]^2, \quad
    \MSE_f = \frac{1}{N} \sum_{i=1}^N \left[f \left(t_i, x_i \right) \right]^2.
\end{equation}

As a concrete example, we consider linear heat equation
\begin{equation}
    u_t - \lambda u_{xx} - g(t, x) = 0
\end{equation}
with one scalar sought-for parameter \(\lambda\) and given source function
\(g(t, x) \).

As a more difficult example, we consider viscous Burgers’ equation    
\begin{equation}
    u_t + \lambda_1 u u_x - \lambda_2 u_{xx} = 0, \quad x\in[-1; 1], t\in[0, 1]
\end{equation}
with sought-for parameter \(\lambda = \left( \lambda_1, \lambda_2 \right)^\T \).
This equation is nonlinear and serves as a prototype of the governing
equations of fluid dynamics. It is known that the solutions of this equation
may develop sharp gradients in finite time even for smooth initial condition.
\end{document}